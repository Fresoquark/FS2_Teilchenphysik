\section{Übereinstimmung Daten und Simulation}
\label{sec:aufgabe6}

Grundlage für Messungen neuer Physik ist immer eine sinnvolle Übereinstimmung 
der Daten mit der Simulation, vor allem in einem Phasenraum, in dem der 
Untergrund dominiert. Dafür werden zunächst die erwartete Anzahl der Events 
berechnet, welche zu einer integierten Luminosität von 
$\mathcal{L} = \SI{1}{\femto\barn}^{-1}$ korrespondieren. Mit Hilfe der Formel 

\begin{equation}
N = \mathcal{L} \sigma A \epsilon
\label{eqn:erwartung}
\end{equation}

kann diese Anzahl berechnet werden. Dabei entspricht der Koeffizient 
$\epsilon * A$ den in Kapitel \label{sec:aufgabe3}
berechneten Werten. Der Wirkungsquerschnitt der Prozesse ist zusammen mit 
den erwarteten Events in Tabellle \ref{tab:Erwartungen} aufgelistet. Die erwarteten 
Events sind für das ausgewählte Sample \texttt{data.mu.2.root} aufgelistet.

\begin{table}
    \centering
    \caption{Anzahl erwarteter Ereignisse für die einzelnen Prozesse im sample \texttt{data.mu.2.root}. Angegeben 
    ist der zugehörige Wirkungsquerschnitt der für die Berechnung der einzelnen Werte nach Formel 
    \eqref{eqn:erwartung} benötigt wird.}
    \label{tab:Erwartungen}
    \begin{tabular}{c|cc}
    \toprule 
    Prozess & $\text{N}_\text{expected}$ & $\sigma$ / $\SI{}{\pico\barn}$ \\
    \midrule
    \texttt{ttbar}      &  3.0282   & 252.82    \\
    \texttt{singletop}  &  3.3576   & 52.47     \\
    \texttt{diboson}    &  2.9967   & 29.41     \\
    \texttt{wjets}      &  6.3202   & 2516.20   \\
    \texttt{zjets}      &  51.1618  & 36214     \\
    \texttt{zprime400}  &  103.4000 & 1.1e2     \\
    \texttt{zprime500}  &  77.0800  & 8.2e1     \\
    \texttt{zprime750}  &  18.8000  & 2.0e1     \\
    \texttt{zprime1000} &  51.7000  & 5.5       \\
    \texttt{zprime1250} &  17.8600  & 1.9       \\
    \texttt{zprime1500} &  7.8020   & 8.3e-1    \\
    \texttt{zprime1750} &  2.8200   & 3.0e-1    \\
    \texttt{zprime2000} &  1.3160   & 1.4e-1    \\
    \texttt{zprime2250} &  0.0630   & 6.7e-2    \\
    \texttt{zprime2500} &  0.0330   & 3.5e-2    \\
    \texttt{zprime3000} &  0.0113   & 1.2e-2    \\
    \bottomrule 
    \end{tabular}
\end{table}


\begin{table}
    \centering
    \caption{Berechnete Gewichte für die einzelnen Prozesse. Der für die Berechnung notwendige 
    Wirkungsquerschnitt ist in Tabelle \ref{tab:Erwartungen} aufgelistet. Die integrierte 
    Luminosität beträgt $\mathcal{L} = \SI{1}{\femto\barn}^{-1}$.}
    \label{tab:Gewichte}
    \begin{tabular}{c|cc}
    \toprule 
    Prozess & $\text{N}_\text{MC}$ & Gewichtung $w$ \\
    \midrule
    \texttt{ttbar}      & 7847944    & 0.03221     \\
    \texttt{singletop}  & 1468942    & 0.03572     \\
    \texttt{diboson}    & 922521     & 0.03188     \\
    \texttt{wjets}      & 66536222   & 0.54428     \\
    \texttt{zjets}      & 37422926   & 0.06724     \\
    \texttt{zprime400}  & 100000     & 1.10000     \\
    \texttt{zprime500}  & 100000     & 0.82000     \\
    \texttt{zprime750}  & 100000     & 0.20000     \\
    \texttt{zprime1000} & 100000     & 0.55000     \\
    \texttt{zprime1250} & 100000     & 0.19000     \\
    \texttt{zprime1500} & 100000     & 0.08300     \\
    \texttt{zprime1750} & 100000     & 0.03000     \\
    \texttt{zprime2000} & 100000     & 0.01400     \\
    \texttt{zprime2250} & 100000     & 0.00067     \\
    \texttt{zprime2500} & 100000     & 0.00035     \\
    \texttt{zprime3000} & 100000     & 0.00012     \\
    \bottomrule 
    \end{tabular}
\end{table}
