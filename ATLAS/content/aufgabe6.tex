\section{Übereinstimmung Daten und Simulation}
\label{sec:aufgabe6}

Grundlage für Messungen neuer Physik ist immer eine sinnvolle Übereinstimmung 
der Daten mit der Simulation, vor allem in einem Phasenraum, in dem der 
Untergrund dominiert. Dafür werden zunächst die erwartete Anzahl der Events 
berechnet, welche zu einer integierten Luminosität von 
$\mathcal{L} = \SI{1}{\femto\barn}^{-1}$ korrespondieren. Mit Hilfe der Formel 

\begin{equation}
N = \mathcal{L} \sigma A \epsilon
\end{equation}

kann diese Anzahl berechnet werden. Dabei entspricht der Koeffizient 
$\epsilon * A$ den in Kapitel \label{sec:aufgabe3}
berechneten Werten. Der Wirkungsquerschnitt der Prozesse ist zusammen mit 
den erwarteten Events in Tabellle \ref{tab:Erwartungen} aufgelistet.