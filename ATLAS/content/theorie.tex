\section{Theoretische Grundlagen}

In dem Lehrstuhlversuch \textit{Search for $t\bar{t}$ resonance with ATLAS detector} wird ein Datensatz, welcher bei 
einer Schwerpunktsenergie von $\sqrt{s} = \SI{8}{\tera\electronvolt}$, entsprechend einer Luminosität von 
$\mathcal{L} = \SI{1}{\femto\barn}^{-1}$,der 2012 am ATLAS Detektor aufgenommen wurde, auf $Z^\prime$-Resonanzen untersucht. 
Suchen nach diesem neuen massiven Teilchen beginnen oft bei Massenskalen von $\SI{500}{\giga\electronvolt}$ und aufwärts. Die 
Skala für neue Physik wird in dem meisten Fällen um die $\SI{1}{\tera\electronvolt}$ gesetzt. Aktuelle Limits auf die 
$Z^\prime$ Resonanz schließen Massen kleiner als XXX aus. \par 

In dieser Analyse wird der mögliche Zerfall des $Z^\prime$ in ein Top-Quark und ein Anti-Top-Quark untersucht. Das Top-Quark 
ist das schwerste bekannte Quark und somit sensitiv auf neue Physik. Am LHC wird es hauptsächlich durch Gluonfusion 
produziert, wohingegen an Beschleunigern mit geringerer Schwerpunktsenergie wie das Tevatron, hauptsächlich Quark-Antiquark 
Annihilation für die Produktion verantwortlich ist. Top-Quarks 
zerfallen fast ausschließlich in ein Bottom-Quark zusammen mit einem $W$-Boson. Letztere können bei der 
Top-Quark Paarproduktion entweder semileptonisch, 
leptonisch oder hadronisch zerfallen. Der leptonische Zerfall beschreibt den Endzustand mit einem geladenen Lepton und 
dem zugehörigen Neutrino für beide Eichbosonen. Der hadronische Zerfall beschreibt den Zerfall beider $W$-Bosonen in 
jeweils zwei Quarks. Der semileptonische Zerfall beschreibt dann ein hadronisch zerfallendes $W$-Boson und ein leptonisch 
zerfallendes. Untersuchungen des leptonischen Zerfalls haben den Nachteil, dass duch die Neutrinos ein hoher Anteil fehlender 
Energie in der Analyse untersucht werden muss, wohingegend die Analyse des hadronischen Zerfalls den Nachteil vieler Jets 
hervorruft. In diesem Versuch wird demnach der semileptonische Zerfall untersucht, der auch \textit{lepton + jets} genannt wird, 
da sowohl ein geladenes Lepton und fehlende Energie verlangt wird, als auch mindestens 4 Jets, die durch den hadronischen $W$-Zerfall und die Bottom-Quarks 
aus dem Top-Quark-Zerfall stammen. \par 

Die Signaturen der untersuchten Objekte im ATLAS Detektor sind wie folgt. Das Muon interagiert im Detektor zunächst als \textit{Minimal Ionizing Particle}, 
einem sogenannten MIP. Es hinterlässt somit weder im Spurdetektor noch in den Kalorimetern eine Signatur. Lediglich in den Muonkammern deponiert es 
Energie. Elektronen werden in den Trackingdetektoren nach ihrer Ladung gekrümmt und deponieren anschließend im elektromagnetischen Kalorimeter ihre 
Energie. Die Quarks aus dem hadronischen Zerfall hadronisieren und schauern hauptsächlich im hadronischen Kalorimeter. Neutrinos sind nur über fehlende 
Energie der bereits rekonstruierten Endzustandsteilchen bestimmbar.

\section{Analysestrategie}