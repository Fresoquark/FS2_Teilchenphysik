\subsection{Erweiterte Eventselektion}
\label{sec:aufgabe3}

Selektiert wird auf den \texttt{lepton+jets} Kanal. Das bedeutet, es wird genau 
ein geladenes Lepton aus dem leptonischen Zerfall eines $W$-Bosons gefordert. 
Dieses Lepton, soll einen Transversalimpuls von mindestens $\texttt{lep$\_$pt} \textgreater 
\SI{50}{\giga\electronvolt}$ besitzen. Leptonen mit geringerem Transversalimpuls 
werden verworfen. In dem leptonischen Verfall, wird zudem ein 
Neutrino produziert. Daher wird eine fehlende Transversalenergie von 
$\texttt{met$\_$et} \textgreater \SI{40}{\giga\electronvolt}$ gefordert. \par 
Da beide Top-Quarks in ein $W$-Boson mit einem Bottom-Quark zerfallen, entstehen
dadurch bereits zwei Quarks. Aus dem hadronischen Zerfall des zweiten $W$-Bosons 
entstehen zwei weitere Quarks, weswegen in der Selektion Events mit 
$\texttt{jet$\_$n} \textless 4$ verworfen werden. Von diesen Jets, werden 
mindestens zwei Jets verlangt, die b-tagged sein müssen. Das wird dadurch erreicht,
indem ein \texttt{MV1} Wert größer als $0.7892$ gefordert wird. Weiterhin soll 
jeder Jet mindestens einen Transversalimpuls von $\texttt{jet$\_$pt} 
\textgreater \SI{100}{\giga\electronvolt}$ besitzen. \par 
Die Pseudorapidität ist sowohl eine Detektorkomponente als auch eine 
Teilcheneigenschaft, denn Teilchenmasse und -energie beeinflussen den 
Abstrahlungswinkel. Der ATLAS Detektor kann innerhalb eines Akzeptanzbereichs 
von $|{\eta}| \textless 2.4$ messen. Alles was diesen Bereich übersteigt, 
kann nicht von den Detektorkomponenten erfasst werden. Deswegen wird auf 
diese Eigenschaft ebenfalls selektiert. Schnitte auf Teilcheneigenschaften 
wie Transversalimpuls sind notwendig um Untergründe zu reduzieren. Da in diesem 
Versuch ein Teilchen mit hoher invarianter Masse untersucht wird, ist es 
sinnvoll hohe Grenzen auf den Transversalimpuls zu setzen. \par 

Die erwarteten Untergründe für diesen Zerfallskanal sind bereits in 
Kapitel \ref{sec:strategie} gelistet. \par

Zur Berechnung des Koeffizienten $\epsilon * A$, welches für Effizienz mal die 
Detektorakzeptanz steht, wird das Verhälnis der Anzahl der Events, die einen 
Cut überstehen, 
mit der Gesamtanzahl der Events nach der kompletten Selektion gebildet. Dadurch 
ergeben sich für die Samples die in Tabelle \ref{tab:Effizienzen} abgebildeten 
Effizienzen.

\begin{table}
    \centering
    \caption{Berechte Effizienzen für die einzelnen Cuts der Eventselektion}
    \label{tab:Effizienzen}
    \resizebox{\textwidth}{!}{\begin{tabular}{c|ccccccc}
    \toprule 
    Sample & $\texttt{lep$\_$n} != 1$ & $\texttt{lep$\_$pt} \textgreater 
\SI{50}{\giga\electronvolt}$ & $\texttt{jet$\_$n} \textgreater 4 $ & $\texttt{jet$\_$pt} \textgreater 
\SI{100}{\giga\electronvolt}$ &  $\texttt{btagged} \textgreater 2$ & $\texttt{met$\_$et} \textgreater 
\SI{50}{\giga\electronvolt}$ &
$\texttt{jet$\_$eta} \textless |2.4|$ \\
    \midrule 
    \texttt{ttbar.el}      & 0.9316 & 0.5071  & 0.2442 & 0.1406  & 0.0605      & 0.0428         & 0.0400 \\
    \texttt{ttbar.mu}      & 0.9200 & 0.4821  & 0.2328 & 0.1340  & 0.0579      & 0.0409         & 0.0382 \\
    \texttt{singletop.el}  & 0.9806 & 0.4323  & 0.0532 & 0.0311  & 0.0103      & 0.0071         & 0.0064 \\
    \texttt{singletop.mu}  & 0.9788 & 0.4104  & 0.0498 & 0.0288  & 0.0097      & 0.0068         & 0.0062 \\
    \texttt{diboson.el}    & 0.9044 & 0.3813  & 0.0119 & 0.0048  & 0.0001      & 7.97e-05       & 7.9691e-05 \\
    \texttt{diboson.mu}    & 0.8731 & 0.3510  & 0.0091 & 0.0038  & 9.99691e-05 & 7.38902e-05    & 6.51973e-05 \\
    \texttt{wjets.el}      & 0.9999 & 0.1709  & 0.0029 & 0.0017  & 1.99407e-05 & 1.39121e-05    & 1.28301e-05 \\
    \texttt{wjets.mu}      & 0.9999 & 0.1589  & 0.0027 & 0.0016  & 2.33554e-05 & 1.56505e-05    & 1.49282e-05 \\
    \texttt{zjets.el}      & 0.7167 & 0.1814  & 0.0067 & 0.0032  & 0.000135906 & 4.8692e-05     & 4.59184e-05 \\
    \texttt{zjets.mu}      & 0.5015 & 0.1256  & 0.0025 & 0.0014  & 7.37231e-05 & 3.18146e-05    & 3.00259e-05 \\
    \texttt{zprime400.el}  & 0.9308 & 0.4464  & 0.1640 & 0.0470  & 0.0192      & 0.0124         & 0.0119 \\
    \texttt{zprime400.mu}  & 0.9201 & 0.4215  & 0.1597 & 0.0539  & 0.0203      & 0.0137         & 0.0131 \\
    \texttt{zprime500.el}  & 0.9324 & 0.5616  & 0.2539 & 0.1499  & 0.0625      & 0.0417         & 0.0399 \\
    \texttt{zprime500.mu}  & 0.9224 & 0.5353  & 0.2319 & 0.1357  & 0.0581      & 0.0389         & 0.0378 \\
    \texttt{zprime750.el}  & 0.9237 & 0.6705  & 0.3881 & 0.3503  & 0.1573      & 0.1216         & 0.1155 \\
    \texttt{zprime750.mu}  & 0.9133 & 0.6451  & 0.3710 & 0.3332  & 0.1526      & 0.1220         & 0.1173 \\
    \texttt{zprime1000.el} & 0.9339 & 0.7342  & 0.4599 & 0.4399  & 0.2087      & 0.1789         & 0.1728 \\
    \texttt{zprime1000.mu} & 0.9182 & 0.7055  & 0.4573 & 0.4330  & 0.2021      & 0.1681         & 0.1596 \\
    \texttt{zprime1250.el} & 0.9355 & 0.7475  & 0.4811 & 0.4665  & 0.2157      & 0.1898         & 0.1820 \\
    \texttt{zprime1250.mu} & 0.9246 & 0.7319  & 0.4867 & 0.4702  & 0.2217      & 0.1931         & 0.1853 \\
    \texttt{zprime1500.el} & 0.9401 & 0.7711  & 0.5014 & 0.4905  & 0.2121      & 0.1905         & 0.1819 \\
    \texttt{zprime1500.mu} & 0.9287 & 0.7483  & 0.5012 & 0.4902  & 0.2125      & 0.1893         & 0.1805 \\
    \texttt{zprime1750.el} & 0.9473 & 0.7758  & 0.5140 & 0.5036  & 0.2146      & 0.1932         & 0.1840 \\
    \texttt{zprime1750.mu} & 0.9383 & 0.7571  & 0.5036 & 0.4914  & 0.2091      & 0.1898         & 0.1819 \\
    \texttt{zprime2000.el} & 0.9451 & 0.7629  & 0.4988 & 0.4873  & 0.1913      & 0.1747         & 0.1683 \\
    \texttt{zprime2000.mu} & 0.9406 & 0.7611  & 0.5171 & 0.5066  & 0.1968      & 0.1799         & 0.1717 \\
    \texttt{zprime2250.el} & 0.9450 & 0.7451  & 0.4948 & 0.4839  & 0.1846      & 0.1688         & 0.1588 \\
    \texttt{zprime2250.mu} & 0.9448 & 0.7639  & 0.5183 & 0.5085  & 0.2012      & 0.1824         & 0.1738 \\
    \texttt{zprime2500.el} & 0.9480 & 0.7509  & 0.4958 & 0.4809  & 0.1759      & 0.1598         & 0.1523 \\
    \texttt{zprime2500.mu} & 0.9433 & 0.7595  & 0.5211 & 0.5078  & 0.1933      & 0.1765         & 0.1669 \\
    \texttt{zprime3000.el} & 0.9437 & 0.7151  & 0.4510 & 0.4253  & 0.1648      & 0.1458         & 0.1368 \\
    \texttt{zprime3000.mu} & 0.9398 & 0.7314  & 0.4833 & 0.4631  & 0.1781      & 0.1607         & 0.1516 \\
    \bottomrule 
    \end{tabular}}
\end{table}

Nach der Eventselektion ist Top-Quark Paarproduktion der größte Untergrund, da die Signatur dieses 
Untergrundes der selektierten Signatur ähnelt.