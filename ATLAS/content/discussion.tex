\section{Diskussion}

Die Ergebnisse dieses Versuchen zeigen einen Hinweis auf einen statistischen Fehler
entweder in der Monte Carlo Simulation selber oder in der Selektion hin. Wird der 
Fehler auf den Bin nach Poisson berechnet, wird festgestellt, dass die Abweichungen 
allerdings noch im Bereich des Fehlers aufzufinden sind. \par

Des Weiteren ist festzustellen, dass die Diskriminanzanalyse gut funktioniert hat, 
da bei der Limitsetzung Massenhypothesen für $m_{Z^\prime} \textless \SI{1250}{\giga\electronvolt}$ 
ausgeschlossen werden können. \par

Zur Bewertung und Beurteilung des Lehrstuhlversuchs ist folgendes Anzumerken. Zu einem
wurden eine Methode der Datenanalyse aufgezeigt, was zu einem sehr interessant war, aber
zum anderen auch sehr anspruchsvoll. Der Versuch ist nur mit weiterführenden Kenntnissen
in die Programmiersprachen \texttt{C++} und \texttt{Root} zu bewältigen und fällt deutlich
schwerer, wenn diese nicht in der Tiefe vorhanden sind. Des Weiteren ist der Umfang des
Versuchs unserer Meinung nach etwas zu hoch. Auf der einen Seite, ist es uns verständlich,
dass nur die volle Analyse das komplette Verständnis für diesen Fachbereich liefert,
allerdings ist es im Hinblick darauf, dass noch weitere Versuche durchgeführt werden
müssen, zeitlich nur schwer machbar in diesem Umfang.
Hinzu kommt die aktuelle Situation, die es einem nur ermöglicht remote miteinander zu arbeiten oder Fragen zu stellen.
Dies hat die Zusammenarbeit und das Verständnis manchmal sehr erschwert.
