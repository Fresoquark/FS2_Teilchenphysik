\section{Diskussion}

Die Ergebnisse dieses Versuchen zeigen einen Hinweis auf einen systematischen Fehler 
entweder in der Monte Carlo Simulation selber oder in der Selektion hin. Die Datenpunkte 
in den Stacked Plots, wie beispielsweise in Abbildung \ref{fig:stackedex} zeigen deutliche 
Streuung von den gestapelten Untergründen. Auch in der Finalen Diskriminanten sind diese 
Abweichungen zu finden, weswegen alle späteren Ergebnisse eine Fortpflunzung dieses 
systematischen Fehlers beinhalten und somit an Relevanz verlieren. \par 

Zur Bewertung und Beurteilung des Lehrstuhlversuchs ist folgendes Anzumerken. Zu einem 
wurden eine Methode der Datenanalyse aufgezeigt, was zu einem sehr interessant war, aber 
zum anderen auch sehr anspruchsvoll. Der Versuch ist nur mit weiterführenden Kenntnissen 
in die Programmiersprachen \texttt{C++} und \texttt{Root} zu bewältigen und fällt deutlich 
schwerer, wenn diese nicht in der Tiefe vorhanden sind. Des Weiteren ist der Umfang des 
Versuchs unserer Meinung nach etwas zu hoch. Auf der einen Seite, ist es uns verständlich, 
dass nur die volle Analyse das komplette Verständnis für diesen Fachbereich liefert,
allerdings ist es im Hinblick darauf, dass noch weitere Versuche durchgeführt werden 
müssen, zeitlich nur schwer machbar in diesem Umfang.