\subsection{Beta-Strahlung}

Verwendet wird ein $^{90}\text{Sr}$ Strahler, der $\beta$-Teilchen mit einer Energie 
von ungefähr $\SI{0.546}{\mega\electronvolt}$ emittiert. Der Strahler ist ein 
reiner $\beta$-Strahler und ein $\beta$-Zerfall erfolgt nach dem Schema

\begin{align*}
    n \to p + e^{-} + \bar{\nu} \, .
\end{align*}

Der weitere Zerfall kann wie folgt beschrieben werden:

\begin{align*}
^{90}\text{Sr} \to ^{90}\text{Y} \to ^{90}\text{Zr} \. .
\end{align*}

Yittrium selber zerfällt über einen reinen beta-Zerfall in Zirkonium und emittiert 
dabei Teilchen mit einer Energie von $\SI{2.28}{\mega\electronvolt}$.
In dieser Größenordnung kann erkannt werden, dass Yittrium und Zirkonium Teilchen 
mit annährend gleicher Energie emittieren. Somit beeinflusst der Zerfall von 
Yittrium in Zirkonium auch die Messung. Beim Beta-Zerfall teilt sich die Energie 
zwischen dem Elektron, dem Neutrino und dem Atomkern auf. Das ionisierende Teilchen
bei einem Beta-Zerfall ist das Elektron, weswegen dessen Wechselwirkungsmechanismen 
mit Materie erläutert werden müssen.

\subsubsection{Wechselwirkungen Elektronen mit Materie}
Elektronen können auf verschiedene Art und Weisen mit Materie interagieren. Die 
Häufigkeit dieser Wechselwirkungen hängt von der kinetischen Energie der 
Elektronen ab. Mögliche Wechselwirkungen beinhalten unter anderem die 
Bremsstrahlungl, Cherenkov Strahlung und
Stöße der Elektronen mit entweder dem Atomkern oder den Hüllenelektronen.
Bremsstrahlung entspricht der elektromagnetischen Strahlung, die beschleunigte Ladung, 
hier das Elektron, im Coulombfeld eines Kerns, abstrahlt. Diese wird allerdings 
erst ab Energien mehrerer $\SI{10}{\mega\electronvolt}$ \cite{DESY} relevant. 
Da die Elektronen der $^{90}\text{Sr}$ im Allgemeinen eine deutlich geringere 
Energien haben, wird Bremsstrahlung für diesen Versuch nicht beachtet. Effekte 
durch Cherenkov Strahlung, also die Strahlung die entsteht, wenn geladene Teilchen
durch ein Medium mit $v_\text{Teilchen} \textgreater c_{Medium}$ propagieren, werden 
aus gleichen Gründen nicht betrachtet. Auf Grund großer Kernmassen, kann auch die 
Wechselwirkung von Elektronen mit den Atomkernen des durchlaufenden Materials 
vernachlässigt werden. \par \medskip
Die relevante Wechselwirkung in diesem Energiebereich ist die Wechselwirkung 
der Elektronen mit den Hüllenelektronen des Materials. Wird dabei das Elektron 
aus der Atomhülle gelöst, wird dieses Sekündärteilchen als Delta-Elektron 
bezeichnet. Der Energieverlust eines Elektrons pro Wegstecke wird über die 
Bethe-Bloch-Gleichung definiert:

\begin{equation}
    - \frac{\symup{d}E}{\symup{d}x} = 2 \pi \text{N}_\text{a} \text{m}_\text{e} c \rho \frac{\text{Z}}{\text{A} \beta^2}
    \left[\ln{\frac{\tau^2 \left(\tau + 2 \right)}{2 \left(I / \text{m}_{\text{e} c^2} \right)^2}}   
    + \text{F}\left(\tau \right) - \delta - 2 \frac{\text{C}}{\text{Z}} \right] \, .
\label{eqn:bloch}
\end{equation}

Die einzelnen Parameter und ihre Bedeutung sind in Abbildung \ref{fig:TabelleBloch} dargestellt.

\begin{figure}
  \centering
  \includegraphics[width=\textwidth]{content/graphics/Bloch.png}
  \caption{Erklärung und Bedeutung der in Formel \eqref{eqn:bloch} verwendeten 
  Parameter zur Bestimmung des Energieverlusts pro Wegstecke von Elektronen 
  bei Durchgang durch ein Medium \cite{Anleitung}.}
  \label{fig:TabelleBloch}
\end{figure}

Der Parameter $\tau$ beschreibt die kinetische Energien und besitzt die Einheit $\text{m}_\text{e} c^2$. 
Die Funktion $ \text{F}\left(\tau \right)$ ist ebenfalls genau bestimmt. 
Die Bethe-Bloch-Gleichung ist eine materialspefizische Gleichung, die nicht nur 
von der Energie des einfallenden Teilchens abhängt, sondern auch von den durch das
Material vorgegebenen Materialgegebenheiten. Bei der Berechnung eines 
einfallenden Teilchens aus dem $^{90}\text{Sr}$ Primärzerfall, mit maximaler 
Zerfallsenergie, ergibt sich eine durchschnittliche Energiedeposition pro 
Wegstrecke in reinem Silizium von $\SI{3.88}{\mega\electronvolt\per\centi\meter}$.

\subsubsection{Energieverteilung in einem Silizium-Sensor}

Bei der Betrachtung der Energieverteilung eines Elektrons in einem 
Siliziumsensors müssen einige mathematische Überlegungen gemacht werden. Der 
zentrale Grenzwertsatz besagt, dass die Summe vieler identisch verteilter
Zufallsvariablen mit endlicher Varianz, näherungsweise einer Gaußverteilung 
entspricht \cite{Grenzwert}. Übertragen auf das Problem, bedeutet das, dass eine 
Normalverteilung für das Energiespektrum der deponierten Energie von den 
geladenen Teilchen für eine ausreichende Dicke des Detektors zu erwarten ist. 
Allerdings ist das hier nicht der Fall, da die Streifen selber sehr dünn sind 
$\left(\SI{300}{\micro\meter}  \right)$ und es somit zu wenigen Stoßprozessen 
kommt. Ein wichtiger Faktor dabei ist, dass die sekundär produzierten 
Delta-Elektronen auf Grund der geringen Dicke des Sensors möglicherweise nicht 
vollständig abgebremst werden und somit nicht die komplette Energie des Primärteilchens 
deponiert wird. Es ergibt sich somit eine Energieverteilung, die asymmetrisch 
verteilt ist, und ihren Peak näher an kleineren Energien finden und Ausläufer 
zu hohen Energien aufweist, eine sogenannte Landauverteilung. \par \smallskip 

Ebenfalls nicht zu vernachlässigen ist die statistische Energieverteilung der 
Beta-Teilchen. Diese beeinflusst das Energiespektrum, wodurch eine Faltung 
der Gaußverteilung mit der Landauverteilung die Beste Beschreibung für das 
Problem liefert.
Eine Charakteristik der Landauverteilung ist, dass der wahrscheinlichste Wert, 
also der Ort des Peakes, näher an kleineren x-Werten (hier also Energien) liegt, 
als der Mittelwert der Verteilung. Dieser Fakt stammt aus der Asymmetrie 
der Verteilung. In diesem Versuch werden keine Energien direkt gemessen. Gemessen 
wird zunächst in Einheiten der Ausgabe des Analog-zu-Digital-Konverters, sogenannten ADC. 
Die ADC Counts können dann mit dem Wissen, dass die Energie zur Erstellung eines 
Elektron-Loch-Paares in Silizium $\SI{3.6}{\electronvolt}$ beträgt, in 
Energien umgerechnet werden.