\subsection{Beta-Strahlung}

Verwendet wird ein $^{90}\text{Sr}$ Strahler, der $\beta$-Teilchen mit einer Energie 
von ungefähr $\SI{0.546}{\mega\electronvolt}$ emittiert. Der Strahler ist ein 
reiner $\beta$-Strahler und ein $\beta$-Zerfall erfolgt nach dem Schema

\begin{align*}
    n \to p + e^{-} + \bar{\nu} \, .
\end{align*}

Der weitere Zerfall kann wie folgt beschrieben werden:

\begin{align*}
^{90}\text{Sr} \to ^{90}\text{Y} \to ^{90}\text{Zr} \. .
\end{align*}

Yittrium selber zerfällt über einen reinen beta-Zerfall in Zirkonium und emittiert 
dabei Teilchen mit einer Energie von $\SI{2.28}{\mega\electronvolt}$.
In dieser Größenordnung kann erkannt werden, dass Yittrium und Zirkonium Teilchen 
mit annährend gleicher Energie emittieren. Somit beeinflusst der Zerfall von 
Yittrium in Zirkonium auch die Messung. Beim Beta-Zerfall teilt sich die Energie 
zwischen dem Elektron, dem Neutrino und dem Atomkern auf. Das ionisierende Teilchen
bei einem Beta-Zerfall ist das Elektron, weswegen dessen Wechselwirkungsmechanismen 
mit Materie erläutert werden müssen.

\subsubsection{Wechselwirkungen Elektronen mit Materie}