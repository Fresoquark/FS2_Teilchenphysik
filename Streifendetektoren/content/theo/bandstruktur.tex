\subsection{Die Bandstruktur}
\label{sec:bandstruktur}

Um Halbleiter zu verstehen, muss zunächst einmal der Ursprung der Bandstruktur
erklärt werden. Ein Elektron, welches sich in einem gebundenen Zustand an einem 
Atom befindet, befindet sich auf einem Energieniveau. Am energetisch günstigsten
ist es für das Elektron im Ruhezustand, das niedrigste Niveau zu besetzen. Um auf
ein höheres Niveau zu kommen, muss Energie zugeführt werden. Die einzelnen Bahnen
eines Atoms representieren dabei jeweils ein Energieniveau. \par

Wird nun ein Material betrachtet, in dem mehrere Atome in unmittelbarer
Nähe zueinander plaziert sind, beeinflussen sich die Potentialtöpfe bzw. die
Energieniveaus untereinader und eine Aufspaltung tritt auf. In Abbildung
\ref{fig:bandstruktur} ist dies für zwei Atome schematisch dargestellt.


\begin{figure}
  \centering
  \includegraphics[width=0.7\textwidth]{content/graphics/bandstruktur.png}
  \caption{Erklärung zur Entstehung von Energiebändern. Bei der Linken Darstellung
  muss angemerkt werden, dass nach dem Pauli Prinzip verboten ist, dass sich
  zwei Teilchen mit exakt gleichen Quantenzahlen auf dem gleichen Energieniveau
  befinden. Dies soll der Potentialtopf auch nicht verdeutlichen, sondern, dass
  jedes Atom für sich bei ausreichend Abstand einen unbeinflussten
  Potentialtopf besitzt \cite{BANDSTRUKTUR}.}
  \label{fig:bandstruktur}
\end{figure}

In einem Material beeinflussen sich allerdings nicht nur zwei Atome, sondern
mehrere, wodurch eine Vielzahl von Aufspaltungen stattfindet. Diese
Aufspaltungen werden dann nicht mehr als diskrete Energieniveaus beschrieben,
sondern können als Energiebänder zusammengefasst werden. Das letzte voll
gefüllte Band wird das Valenzband genannt und das obere, nicht vollständig
besetzte Band, das Leitungsband. Die Bandlücke dazwischen verdeutlicht einen
verbotenen Bereich, da dort keine Energieniveaus oder Zustände vorhanden sind,
die besetzt werden können.
