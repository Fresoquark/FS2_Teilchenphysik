In dem Lehrstuhlversuch \textit{Vermessung und Untersuchung von Silizium-Streifensensoren} soll die Arbeitsweise eines in der Teilchenphysik üblichen Silizium-Halbleiterdetektors untersucht werden.
Dafür wird ein bereits bestehendes Detektorsystem der Firma Alibava Systems genutzt.
Anhand dieses Detektors soll untersucht werden, wie der Nachweis von Teilchenspuren und die Verarbeitung der Daten in teilchenphysikalischen Detektoren passiert.
Der Aufbau des Streifensensors wird durch eine Laserquelle untersucht, damit schlussendlich die genau Vermessung einer Strontium90-Quelle durchgeführt werden kann.

Solche Halbleiterdetektoren finden Verwendung in fast jedem großen, teilchenphysikalischem Experiment.
Der hier untersuchte Detektor wurde für den ATLAS-Detektor entwickelt, welcher sich am LHC am CERN in Genf befindet.
Bei dem ATLAS-Detektor handelt es sich um einen sogenannten $4\pi$-Detektor.
Er deckt somit fast alle Winkelbereiche des Raumwinkels ab und ist zylindrisch um das Strahlrohr aufgebaut.
Der wichtigste Teil des Detektors liegt am Kollisionspunkt, wo die Protonen aufeinander treffen.
Direkt am Strahlrohr liegt dort zunächst der Inner Detector, welcher aus dem Pixeldetektor, dem Silizium-Streifendetektor und dem Übergangsstrahlungsspurdetektor besteht.
Alle diese Detektoren sind schlussendlich für die Verfolgung der Teilchenspuren zuständig.
Dies wird durch die Krümmung der Teilchenbahnen durch einen Magneten noch verbessert.
Durch seinen Aufbau aus kleineren Einheiten (Pixeln) ist der Pixeldetektor zu einer höheren Auflösung fähig als der Siliziumstreifendetektor.
Pixeldetektoren sind jedoch teuer und werden daher nur sehr nahe am Kollisionspunkt verbaut.
Der Siliziumstreifendetektor befindet sich direkt dahinter und besteht aus vier Doppellagen Silizium, die leicht gegeneinander verdreht sind.
Der Siliziumstreifendetektor besteht aus insgesamt 4088 Modulen mit jeweils 768 Streifen.
In diesem Versuch wird ein Detektor mit 128 Streifen genutzt.
