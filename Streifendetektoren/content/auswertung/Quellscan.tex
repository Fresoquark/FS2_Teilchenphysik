\subsection{Der Große Quellscan}
\label{sec:Quellscan}

Dieser Versuchsteil wird mit 1.000.000 Events durchgeführt. Zunächst wird 
in einem Histogramm, die Anzahl der Cluster pro Event mit der Anzahl der 
Kanäle pro Cluster verglichen. Dies ist in Abbildung \ref{fig:AnzahlCluster} 
dargestellt.

\begin{figure}[H]
  \centering
  \includegraphics[width=0.7\textwidth]{plots/clustersamount_hist.pdf}
  \caption{Grafische Darstellung der Anzahl der Kanäle pro Cluster und 
  der Anzahl der Cluster pro Event für einen großen Quellscan einer 
  $^{90}\text{Sr}$ Quelle.}
  \label{fig:AnzahlCluster}
\end{figure}

Zu einem ist erkennbar, dass meistens ein Kanal pro Cluster beteiligt ist. Die 
Häufigkeit der beteiligten Kanäle pro Cluster wird geringer für größer werdende 
Kanalanzahl. Des Weiteren ist erkennbar, dass pro Event meistens ein Cluster 
entsteht. Bei einigen Events entstehen auch zwei Cluster oder kein Cluster, wobei 
diese Häufigkeit geringer ist gegenüber der Häufigkeit für ein Cluster pro Event.
Für drei oder mehr Cluster ist die Häufigkeit verschwindend gering.
\par \medskip 

Als nächstes wird eine Hitmap erstellt, in der die Anzahl der Ereignisse pro 
Kanal dargestellt wird. Die Hitmap ist in Abbildung \ref{fig:Hitmap} 
dargestellt.

\begin{figure}[H]
  \centering
  \includegraphics[width=0.7\textwidth]{plots/hitmap.pdf}
  \caption{Grafische Darstellung Anzahl der Ereignisse pro Kanal.}
  \label{fig:Hitmap}
\end{figure}

Zu erkennen ist eine große Beteiligung der Streifen von 35 bis 60, da dort 
ein Maximum zu sehen ist. Streifen von 0 bis 35 zeigen auch eine hohe Beteiligung 
und Streifen von 60 bis 128 eine immer geringer werdende. Im Vergleich zu 
Abbildung \ref{fig:lazorchannel} ist zu erkennen, dass bei einem großen Quellscan 
deutlich mehr Streifen beteiligt sind, was an der großen Ausdehnung und größeren 
Streuwinkel der Quelle liegen kann. \par \medskip 

Als nächstes wird das Energiespektrum der Quelle ausgemessen. Dazu werden zunächst 
die ADC Werte in Abhängigkeit von ihrer Häufigkeit dargestellt. Es muss beachtet 
werden, dass bei der Datenverarbeitung ADC Counts mit einem Eintrag kleiner als Null 
verworfen wurden. Für die Berechnung des Mittelwerts wurden weiterhin nur Werte, 
welche im Gültigkeitsbereich der Kalibrierungsfunktion \eqref{eqn:kalib} liegen (sprich ADC 
Werte im Bereich von $0 \textless \text{ADC} \textless 254.6729$) liegen, 
betrachtet. Das ADC-Energiespektrum ist in Abbildung \ref{fig:ADCSpektrum} 
dargestellt. 

\begin{figure}[H]
  \centering
  \includegraphics[width=0.7\textwidth]{plots/signal_Quelle.pdf}
  \caption{Signal der $^{90}\text{Sr}$ Quelle angegeben als ADC Count mit 
  der Häufigkeit, mit der diese im großen Quellscan auftauchen.}
  \label{fig:ADCSpektrum}
\end{figure}

Der Mittelwert und der Wahrscheinlichste Wert (MPV: Most probable 
value) betragen:

\begin{align*}
    \bar{\text{ADC}} &= 129.87 \\ 
    \text{MPV} &= 81.92 \, .
\end{align*}

Mit Hilfe der Kalibrierungsfunktion \eqref{eqn:kalib} wird das Spektrum 
dann in ein Energiespektrum überführt. Das keV-Energiespektrum ist in Abbildung 
\ref{fig:Energiespektrum} dargestellt.

\begin{figure}[H]
  \centering
  \includegraphics[width=0.7\textwidth]{plots/signal_Quelle_energie.pdf}
  \caption{Signal der $^{90}\text{Sr}$ Quelle angegeben als Energie mit 
  der Häufigkeit, mit der diese im großen Quellscan auftauchen.}
  \label{fig:Energiespektrum}
\end{figure}

Der Mittelwert und der Wahrscheinlichste Wert (MPV: Most probable 
value) betragen:

\begin{align*}
    \bar{\text{E}} &= \SI{157.61}{\kilo\electronvolt} \\ 
    \text{MPV} &= \SI{93.26}{\kilo\electronvolt} \, .
\end{align*}
