\subsection{Bestimmung der Charge Collection Efficiency}
\label{sec:CCE}
In diesem Versuchsteil wird die Effizienz der Ladungssammlung, oder auch 
Charge Collection Efficiency, zunächst mit einem Laser und dann mit 
einer Quelle bestimmt. Im Anschluss daran werden die beiden Messungen 
verglichen.

\subsubsection{Mittels eines Lasers}
\label{sec:CCEL}

In diesem Teil wird die Charge Collection Efficiency bei einer 
Lasermessung bestimmt. Gemessen werden jeweils 1000 Events 
bei Spannungen von $\SI{0}{}$ bis $\SI{200}{\volt}$ in 
$\SI{10}{\volt}$ Schritten. Das Auswertungsskript liefert für jede 
Spannung eine Datei, in der in jeder Zeile das Signal je 
Kanal gespeichert ist. Für die Datenpunkte in Abbildung 
\ref{fig:CCEL} wird für jede Spannungsdatei über die Signale in 
den Kanälen gemittelt und diese Mittelwerte dann pro Spannung 
aufgetragen. Zur Berechnung des Fits wird Gleichung
\eqref{eqn:Eindringtiefe} verwendet. Dabei ist zu beachten, dass 
die Dicke der Depletionszone $d_C$ unterhalb der Depletionsspannung 
eine nicht konstante Funktion ist. Bei Werten oberhalb und ab der 
Depletionsspannung entspricht die Dicke der Depletionszone natürlicherweise 
der Sensordicke. Dies muss beim Parametrisieren der Funktion 
beachtet werden. Die Daten und die dazu passende Fitfunktion ist in 
Abbildung \ref{fig:CCEL} dargestellt.

\begin{figure}[H]
  \centering
  \includegraphics[width=0.7\textwidth]{plots/CCEL.pdf}
  \caption{Bestimmung der Charge Collection Efficiency mit Hilfe eines 
  Lasers. Die Ausgleichsfunktion wird mit Hilfe von Gleichung 
  \eqref{eqn:Eindringtiefe} berechnet. Ab einer Spannung von $\SI{70}{\volt}$ geht 
  diese in eine konstante Funktion über, auf Grund der Stufenfunktion des 
  Parameters $d_C$.}
  \label{fig:kalibfunktion}
\end{figure}

Verglichen mit der Kennlinie aus Abbilung \ref{fig:kennlinie}, lässt sich 
feststellen, dass die Depletionszone später beginnnt. Die Depletionsspannung 
in diesem Versuchsteil wird daher auf $U_{dep} = \SI{75}{\volt}$ angepasst. Dies 
sind $\SI{5}{\volt}$ mehr als die in \ref{sec:kennlinie} bestimmte 
Depletionsspannung. \par 

Der Parameter $a$ aus Gleichung \eqref{eqn:Eindringtiefe} wird als freier 
Fitparameter eingeführt, um die Eindringtiefe des Lasers herauszufinden. 
Dieser entspricht nach durchführung der Parameterbestimmung mit Hilfe 
von \texttt{scipy.optimize.curve$\_$fit} einem Wert von:

\begin{align*}
    a = \SI[parse-numbers=false]{174.43 \pm 5.8 \times 10^{-5}}{\micro\meter} \, .
\end{align*}
