\subsubsection{Vergleich der beiden Messungen}
\label{sec:Vergleich}

In einem Diagramm werden die beiden Ergebnisse der Charge Collection 
Efficiency für die Laser- und Quellmessung miteinander verglichen 
und aufgetragen. Das Ergebniss ist in Abbildung 
\ref{fig:Vergleich} dargestellt.

\begin{figure}[H]
  \centering
  \includegraphics[width=0.7\textwidth]{plots/CCEQuelle_Laser.pdf}
  \caption{Bestimmung der Charge Collection Efficiency mit Hilfe eines 
  Lasers und einer Quelle. }
  \label{fig:Vergleich}
\end{figure}

Zu erkennen ist, dass die CCE bei der Quelle für einen Spannungswert 
von $\SI{0}{\volt}$ nicht 0 ist, anders als beim Laser. Des Weiteren 
scheint die CCE vom Laser schneller und steiler anzusteigen, weswegen 
diese auch früher in die Sättigung über geht, als die Quelle. Diese zeigt 
einen weniger steilen Anstieg und geht auch erst später in die Sättigung.