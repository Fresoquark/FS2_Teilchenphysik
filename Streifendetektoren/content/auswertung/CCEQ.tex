\subsubsection{Mittels einer Quelle}
\label{sec:CCEQ}

In diesem Teil wird die Charge Collection Efficiency bei einer 
Quellmessung bestimmt. Gemessen werden jeweils 10000 Events 
bei Spannungen von $\SI{0}{}$ bis $\SI{200}{\volt}$ in 
$\SI{10}{\volt}$ Schritten. Das gegebene Auswertungsskript erstellt 
dann für jede Spannung eine Datei, in der in jeder Zeile die 
Einträge stehen, die diesem Cluster zugeordnet werden. Für die Auswertung 
werden die Einträge einer Zeile addiert und danach über alle Zeilen, also 
alle Cluster der Datei gemittelt. Zunächst werden dann die gemittelten ADC 
Werte gegen ihre Spannung aufgetragen, zu sehen in Abbildung \ref{fig:ADCCluster}.

\begin{figure}[H]
  \centering
  \includegraphics[width=0.7\textwidth]{plots/CCEQuelle.pdf}
  \caption{Gemittelte Clustereinträge als ADC Werte gegen die jeweilige 
  Spannung aufgetragen. Die Charge Collection Efficiency wurde hier mit 
  Hilfe einer $^{90}\text{Sr}$ Quelle bestimmt.}
  \label{fig:ADCCluster}
\end{figure}

Im Anschluss daran werden die ADC Einträge mit Hilfe der 
Kalibrierungsfunktion \eqref{eqn:kalib} in Energien umgerechnet. Die Charge 
Collection Efficiency in Abhängigkeit von der Energie ist in 
Abbildung \ref{fig:Energiecluster} dargestellt. Da Effizienzen maximal einen 
Wert von 1 erreichen können, wird in beiden Diagrammen auf den Sättigungswert 
normiert.

\begin{figure}[H]
  \centering
  \includegraphics[width=0.7\textwidth]{plots/CCEQuelleenergie.pdf}
  \caption{Gemittelte Clustereinträge als Energie Werte gegen die jeweilige 
  Spannung aufgetragen. Die Charge Collection Efficiency wurde hier mit 
  Hilfe einer $^{90}\text{Sr}$ Quelle bestimmt und die Energie mit 
  Hilfe der Kalibrierungsfunktion \eqref{eqn:kalib} bestimmt.}
  \label{fig:Energiecluster}
\end{figure}