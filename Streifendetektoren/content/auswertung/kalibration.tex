\subsection{Kalibrationsmessung}
\label{sec:kalibration}

Bei der Kalibrationsmessung sendet die Kontrolleinheit einen wohldefinierten 
Elektronenpuls an den BEETLE Chip, um später ADC Counts in Ladung umrechnen 
zu können.
Zunächst wird eine Delay Messung durchgeführt, die den optimalen Abstand zwischen
Signal und Auslese des BEETLE Chips bestimmen soll. Dazu werden die ADC Counts 
in Abhängigkeit der Zeit gemessen und aufgetragen. Das Maximum der Messwerte wird 
dann für den weiteren Verlauf eingetragen. Die Messwerte für den Delay 
Scan und das ermittelte Maximum sind in Abbildung \ref{fig:delay} 
dargestellt.

\begin{figure}[H]
  \centering
  \includegraphics[width=0.7\textwidth]{plots/delay_messung.pdf}
  \caption{Gemessene ADC Counts in Abhängigkeit von der Zeit. Das 
  Maximum beträgt $t_{max} \approx \SI{48}{\nano\second}$.}
  \label{fig:delay}
\end{figure}

Das Maximum befindet sich an der Position $t_{max} \approx \SI{48}{\nano\second}$
und wird für die weitere Messung als Delay eingetragen. Laut Anleitung 
\cite{Anleitung} liegen typische Werte für einen Delay bei 
ungefähr $t_{max} \approx \SI{65}{\nano\second}$. Die Abweichung wird 
in der Diskussion \ref{sec:diskussion} genauer untersucht. \par \smallskip

Im Anschluss daran, werden für fünf Kanäle Kalibrationskurven aufgenommen. Vier 
davon werden bei einer Spannung von $U = \SI{90}{\volt}$ aufgenommen und eine 
bei einer Spannung von $U = \SI{0}{\volt}$. Die untersuchten Streifen sind 
Streifen 10, 20, 30 und 40, wobei für 40 beide Spannungen gemessen werden. 
Die Kalibrationskurven für $U = \SI{90}{\volt}$ 
sind in Abbildung \ref{fig:kalib} dargestellt. Aufgetragen sind die ADC Counts 
in Abhängigkeit vom Elektronenpuls bzw. der Energie. Um eine Energie in MeV 
auftragen zu können, werden die aufgenommenen Werte zunächst durch $10^6$ geteilt 
und mit $\SI{3.6}{\electronvolt}$ multipliziert, da dies die Energie ist, die benötigt 
wird um ein Elektron-Loch Paar in Silizium zu erzeugen.

\begin{figure*}
    \centering
    \begin{subfigure}[b]{0.475\textwidth}
        \centering
        \includegraphics[width=\textwidth]{plots/calib_channel10.pdf}
    \end{subfigure}
    \hfill
    \begin{subfigure}[b]{0.475\textwidth}  
        \centering 
        \includegraphics[width=\textwidth]{plots/calib_channel20.pdf}
    \end{subfigure}
    \vskip\baselineskip
    \begin{subfigure}[b]{0.475\textwidth}   
        \centering 
        \includegraphics[width=\textwidth]{plots/calib_channel30.pdf}
    \end{subfigure}
    \hfill
    \begin{subfigure}[b]{0.475\textwidth}   
        \centering 
        \includegraphics[width=\textwidth]{plots/calib_channel40.pdf}  
    \end{subfigure}
    \caption{Gemessene ADC Counts in Abhängigkeit des Elektronenpulses für die Streifen 10, 20, 30 und 40 bei einer 
    Spannung von $U = \SI{90}{\volt}$. Bei allen ist ein Sättigungsverhalten ab ungefähr 
    $E \approx \SI{0.55}{\mega\electronvolt}$ erkennbar.}
    \label{fig:kalib}
\end{figure*}

Für den Streifen 40 wird noch ein Vergleichsdiagramm erstellt, in dem der Verlauf 
für $U = \SI{0}{\volt}$ mit dem Verlauf für $U = \SI{90}{\volt}$ 
verglichen wird. Die beiden Kurven sind in Abbildung \ref{fig:kalibvergleich} 
dargestellt.

\begin{figure}[H]
  \centering
  \includegraphics[width=0.7\textwidth]{plots/calib_channel40_comparison.pdf}
  \caption{Gemessene ADC Counts in Abhängigkeit des Elektronenpulses für den Streifen 40 bei einer 
    Spannung von $U = \SI{0}{\volt}$ und $U = \SI{90}{\volt}$ im Vergleich.}
  \label{fig:kalibvergleich}
\end{figure}

Abbildung \ref{fig:kalibvergleich} zeigt, dass der Verlauf für eine Spannung 
$U = \SI{0}{\volt}$ ähnlich ist, die Kurve jedoch flacher ist, als bei einer Spannung 
von $U = \SI{90}{\volt}$. Dies stammt daher, dass bei einer Spannung unterhalb der 
Depletionsspannung noch Rekombination des Elektron-Loch-Paares stattfinden kann, 
welches die Anzahl der gemessenen Counts beeinflusst. \par 

Alle Diagramme aus Abbildung \ref{fig:kalib} und \ref{fig:kalibvergleich} 
zeigen eine Sättigung ab ungefähr $E \approx \SI{0.55}{\mega\electronvolt}$. 
Das bedeutet, dass ab einem gewissen Schwellwert alle Counts den gleichen 
Energiewert zugewiesen bekommen und der Detektor ab dort nicht mehr sensitiv 
messen kann. \par \medskip 

Für spätere Messungen wird eine Kalibrationsfunktion benötigt, um die gemessenen 
ADC Counts in eine Energie umrechnen zu können. Dafür werden von den Streifen 
in Abbildung \ref{fig:kalib} pro Punkt jeweils die Mittelwerte genommen. Diese 
werden dann in einem Energie-ADC Diagramm und mit einem Polynom vierten Grades 
gefittet. Die Messwerte und die Fitfunktion sind in Abbildung 
\ref{fig:kalibfunktion} dargestellt.

\begin{figure}[H]
  \centering
  \includegraphics[width=0.7\textwidth]{plots/calib_mean.pdf}
  \caption{Mittelwerte der Daten aus Diagramm \ref{fig:kalib} für eine 
  Spannung von $U = \SI{90}{\volt}$ im Vergleich grafisch dargestellt 
  und mit Hilfe eines Polynoms 4. Grades gefittet.}
  \label{fig:kalibfunktion}
\end{figure}

Ein sinnvoller Fit war nur bis zu einem ADC-Wert von 254,6729 möglich, da 
ab diesem Punkt die Funktion weder grafisch noch von den Parametern passend war. 
Mit Hilfe von \texttt{scipy.optimize.curve$\_$fit} wurde der Fit durchgeführt und die 
Parameter und dessen Unsicherheiten berechnet .\par 
Für die Funktion: 

\begin{equation}
E \left(ADC  \right) = a + b * ADC + c * ADC^2 + d * ADC^3 + e * ADC^4
\label{eqn:kalib}
\end{equation}

ergeben sich die folgenden Parameter: 

\begin{align*}
    a &= \SI[parse-numbers=false]{0.00865  \pm   6.3*10^{-6}}{\mega\electronvolt} \\
      &= \SI[parse-numbers=false]{8.6477 \pm 0.0063}{\kilo\electronvolt} \\
    b &= \SI[parse-numbers=false]{0.00013  \pm   1.6*10^{-8}}{\mega\electronvolt} \\
      &= \SI[parse-numbers=false]{131.054 \pm 0.016}{\electronvolt} \\
    c &= \SI[parse-numbers=false]{2.0*10^{-5}  \pm   3.5*10^{-12}}{\mega\electronvolt} \\
      &= \SI[parse-numbers=false]{20 \pm 3.5*10^{-6}}{\electronvolt} \\
    d &= \SI[parse-numbers=false]{-1.4*10^{-7}  \pm   1.1*10^{-16}}{\mega\electronvolt} \\
      &= \SI[parse-numbers=false]{-0.14 \pm 1.1*10^{-10}}{\electronvolt} \\
    e &= \SI[parse-numbers=false]{3.5e-10  \pm   3.8*10^{-22}}{\mega\electronvolt} \\
      &= \SI[parse-numbers=false]{3.5e-4 \pm 3.8*10^{-16}}{\electronvolt} \, . \\
\end{align*}
