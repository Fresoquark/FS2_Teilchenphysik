\subsection{Strom-Spannungs-Kennlinie}
\label{sec:kennlinie}

Zunächst muss die Depletionsspannung gemessen werden, da nur ein vollständig 
depletierter Sensor eine zufriedenstellende Messung der Energiedeposition 
eines Teilchen liefern kann. Dies liegt der Tatsache zu Grunde, dass ein 
einfallendes Teilchen in diesem Fall, ein Elektronen-Loch Paar erzeugt, welches 
nicht direkt rekombinieren kann. Der Strom in Abhängigkeit von der Spannung ist 
in Abbildung \ref{fig:kennlinie} dargestellt. Aus dem Diagramm kann eine 
Depletionsspannung von ungefährt $U_{dep} \approx \SI{70}{\volt}$ 
angenommen werden. Die vom Hersteller angegebene Spannung beträgt 
$U_{dep} = 60$ bis $\SI{80}{\volt}$.

\begin{figure}[H]
  \centering
  \includegraphics[width=0.7\textwidth]{plots/kennlinie.pdf}
  \caption{Aufgenommene Strom-Spannungsmesswerte. Aus dem Diagramm wird eine 
  Depletionsspannung von ca. $U_{dep} \approx \SI{70}{\volt}$ angenommen.}
  \label{fig:kennlinie}
\end{figure}

Für den weiteren Verlauf wird eine Spannung von 
$U = \SI{90}{\volt}$ eingestellt. Die Erhöhung der Spannung vom gemessenen 
Wert der Depletionsspannung hat mehrere Gründe. Zu einem, kann so sichergestellt 
werden, dass der Detektor vollständig depletiert ist, da die 
Abschätzung von $U_{dep} \approx \SI{70}{\volt}$ Fehler beinhalten kann. 
Des Weiteren ist anhand der Grafik zu erkennen, dass der Leckstrom 
trotzdem weiterhin steigt und sich im Bereich um die Depletionsspannung 
noch im Wendepunkt befindet. Zudem können thermische und somit statistische 
Effekte die Messung beeinflussen.