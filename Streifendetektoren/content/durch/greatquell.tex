\subsection{Der große Quellscan}

Für den finalen, großen Quellscan wird erneut ein RS Run durchgeführt.
Diesmal mit 1.000.000 Events.
Das mitgelieferte Programm wertet die Rohdaten aus und wendet einen Signal-to-Noise-Cut an.
Es bestimmt die Pedestals, die Noise, den Common Mode und generiert wie zuvor auch die Cluster.
Um die Daten weiter auswerten zu könne ist später eine Umrechnung der ADC Counts in ein entsprechendes $\si{\kilo\electronvolt}$ Energie-Spektrum vorzunehmen.

Zunächst sollen dann die Cluster pro Event und die Kanäle pro Cluster dargestellt werden.
Weiterhin soll für die Übersicht eine Hitmap angefertigt werden.
Die oben erwähnten Spektren werden einmal in ADC Counts und Energie dargestellt um daraus den Mittelwert der Energie und den MPV bestimmen zu können.
