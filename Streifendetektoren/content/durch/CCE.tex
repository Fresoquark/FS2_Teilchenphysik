\subsection{Bestimmung der Charge Collection Efficiency mit Laser und Quelle}

Die Bestimmung der Charge Collection Efficiency (CCE) wird sowohl mit dem Laser als auch mit der Quelle durchgeführt.

Für den Laser wird eines der zuvor bestimmten Maxima genutzt.
Der Laser wird erneut fokussiert.
Die Vorspannung wird in $\SI{10}{\volt}$ Schritten von $\SI{0}{\volt}$ bis $\SI{200}{\volt}$ variiert.
Für jede Stufe werden im Laser Run 1000 Events aufgenommen.
Die so erhaltenen Messpunkte geben schlussendlich Auskunft über die CCE des Detektors.
Durch einen Fit der Daten mittels Gleichung \eqref{eqn:Eindringtiefe} kann die EIndringtiefe des Lasers berechnet werden.

Für die Quelle muss der Aufbau zunächst wieder umgebaut werden.
Der Reiter muss in die Quellenposition gebracht werden.
Die gesamte Detektoreinheit wird dann in einen mit Bleiziegeln ausgekleideten Behälter gelegt.
Die Quelle wird auf die entsprechende Vorrichtung gelegt.
Das Messprogramm an sich verläuft analog zu der des Lasers.
Es wird in $\SI{10}{\volt}$ Schritten von $\SI{0}{\volt}$ bis $\SI{200}{\volt}$ jeweils 10000 Events vermessen.
Damit das Timing korrekt eingestellt ist, muss die Latenz auf 129 eingestellt und der Trigger richtig eingestellt werden.
Der Trigger sollte im OR und Trigger-Pulse-in-Modus betrieben werden.
Der Trigger-Threshold 1 ist auf -50 und Der Trigger-Threshold 2 ist auf 50 einzustellen.
Die Messung wird über den RS Run gestartet.
Das mitgelieferte Programm berechnet aus den Rohdaten zusätzlich die Verteilung der Cluster des Sensors.
Daraus wird die mittlere Clusterenergie in Abhängigkeit der angelegten Soannung bestimmt.
Um die beiden Methoden zu vergleichen, werden diese in einem extra Plot zusammen dargestellt.








%
