Zur Erleichterung der Auswertung wird ein Auswertescript zur Verfügung gestellt, dass die Messdaten in .txt Dateien überschreibt und einige Analyseschritte durchführt.

\subsection{Messung der Strom-Spannungs-Kennlinie}

Wurden die Geräte wie in Abschnitt \ref{sec:aufbau} miteinander verkabelt und das Programm ordnungsgemäß gestartet kann die Strom-Spannungs-Kennlinie aufgenommen werden.
Der Aufbau läuft korrekt, wenn nach dem Start des Programms die rote LED erlischt und lediglich die grüne brennt.
Dafür wir die Strom-Spannungs-Kennlinie in Schritten von $\SI{10}{\volt}$ der Vorspannung von 0 bis $\SI{200}{\volt}$ aufgenommen.
Diese wird mit dem Drehregler eingestellt, dabei muss die Besonderheit für $\SI{0}{\volt}$ berücksichtigt werden.
Damit soll die vom Hersteller angegebene Depletionsspannung von 60 bis $\SI{80}{\volt}$ verifiziert werden.
Für nachfolgende Messungen soll dann, sofern nicht anders angegeben eine um mindestens $\SI{20}{\volt}$ höhere Spannung eingestellt werden.
