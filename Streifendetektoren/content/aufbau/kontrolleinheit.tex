\subsection{Kontrolleinheit}

Die Kontrolleinheit dient der Steuerung der Detektoreinheit und damit des Detektors.
Die Vorspannung kann über einen Drehregler eingestellt werden und der abfließende Leckstrom dort abgelesen werden.
Zu Beachten ist, dass bei $\SI{0}{\volt}$ ein hoher Ausschlag innerhalb des Leckstroms sichtbar ist.
Der Grund ist, dass duch eine leichte Überdrehung des Regler eine Vorspannung in Durchlassrichtung angelegt wird und der Sensor leicht leiten wird.
Der Laser wird ebenfalls hier erzeugt und kann über ein optisches Kabel an die Detektoreinheit geleitet werden.
Weiterhin sind die Anschlüsse für das Triggerkabel hier verbaut.

Die Steuerung kann dann über das Alibava-gui erfolgen.
Dort lassen sich die einzelnen Messprogramme starten und die entsprechenden Parameter, wie zum Beispiel aus den Kalibrationsmessungen, eintragen.
Von dort aus ist es möglich die Daten für eine weitergehende Auswertung abzuspeichern.
