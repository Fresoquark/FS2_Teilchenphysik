\section{Diskussion}
\label{sec:diskussion}

Die Messung der Strom-Spannungs-Kennlinie zeigt eine gute Übereinstimmung mit 
den erwarteten Daten. Herstellerangaben für die Depletionsspannung liegen 
im Bereich zwischen $U_{dep} = 60$ bis $\SI{80}{\volt}$. Mit einer 
Depletionsspannung von $U_{dep} \approx \SI{70}{\volt}$ liegt der 
Messwert also in einem akzeptablen Bereich. Möglicherweise ist die 
Abschätzung auf $U_{dep} \approx \SI{70}{\volt}$ etwas zu hoch und die 
Depletionsspannung liegt um einige wenige Volt geringer, allerdings ist dies 
in der Abbildung \ref{fig:kennlinie} schwer identifizierbar. \par \medskip

Die Untergrundmessung zeigt zu einem, dass ohne externes Feld im Mittel 
ungefähr 4.6 ADC Counts gemessen werden. Zusammen mit dem Common Mode Shift, 
bei dem die Messwerte näherungsweise einem Theoriegauß um die Null verteilt 
entsprechen, ergibt sich ein ungefähres mittleres Gesamtrauschen von 
ungefähr 508 ADC Counts. Ein systematischer Fehler in der Messung ist in diesem 
Versuchsteil auszuschließen. \par \medskip

Bei der Kalibrationsmessung befindet sich das gemessene Maximum im Bereich von 
$t_{max} \approx \SI{48}{\nano\second}$. Laut Anleitung \cite{Anleitung} sollte 
dieses im Bereich von $t_{max} \approx \SI{65}{\nano\second}$ liegen. 
Rückblickend an den Versuchstag gedacht, ist aufgefallen, dass zunächst eine 
Messung durchgeführt wurde, bei der die Spannung noch auf Null Volt lag. Erst 
später ist uns dies aufgefallen und wir haben die Messung für eine 
Depletionsspannung von $\SI{90}{\volt}$ wiederholt. Eine dieser Messung ergab 
ein Maxiumum um die $t_{max} \approx \SI{60}{\nano\second}$, das andere 
war sehr gering. Daher wird vermutet, dass durch einen menschlichen Fehler 
das falsche Spektrum, also das Spektrum bei $\SI{0}{\volt}$ abgespeichert wurde, 
da dort der optimale Delay kleiner war. Für die weiteren Messungen 
wurde allerdings der richtige Wert für den Delay verwendet. \par \smallskip 

Des Weiteren konnte festgestellt werden, dass der Fit mit einem Polynom 4. Grades 
nicht auf dem Gesamten Messbereich möglich war. Daher wurde eine willkürliche 
Grenze gesetzt, bei der der Fit optisch und von den Parametern her, sinnvoll 
möglich war. \par \medskip 

Laut Anleitung hat der Laser eine Ausdehnung auf dem Sensor von 
$d_\text{laser} = \SI{20}{\micro\metre}$. Die von uns bestimmte Ausdehnung 
beträgt einem Wert von $d_\text{laser} = \SI{18.4 \pm 0.6}{\micro\metre}$. 
Die entspricht einer prozentualen Abweichung von $\SI{8}{\percent}$ oder 
einer Sigma-Umgebung. Daher kann dieser Versuchsteil als gut umgesetzt 
bewertet werden. Der gemessene pitch, also der Abstand 
zwischen den Sensorstreifen, entspricht
$\SI{160 \pm 110}{\micro\metre}$. Hierbei ist zu beachten, dass durch die Methode, 
der Bestimmung des Mittleren Abstandes der einzelnen Peaks, keine Unsicherheit 
bestimmbar war, aber davon auszugehen ist, dass diese ein genaueres 
Ergebnis liefern würde. In der Anleitung ist ein 
pitch von $\SI{160}{\micro\metre}$ als Angabe für den Sensor gegeben. Die Werte 
stimmen überein, jedoch kann durch den hohen Fehler bzw. die Abwesenheit eines 
Fehlers keine genauere Bewertung vorgenommen werden. \par \medskip 

Die Messung der Charge Collection Efficiency für den Laser ergibt eine 
mittlere Eindringtiefe von $a = \SI[parse-numbers=false]{174.43 \pm 5.8 \times 10^{-5}}{\micro\meter}$.
Laut Anleitung hängt diese von der Wellenlänge des Lasers ab und beträgt 
für $\SI{960}{\nano\meter}$ $\SI{74}{\micro\meter}$ und für 
$\SI{1073}{\nano\meter}$ $\SI{380}{\micro\meter}$. Der verwendete Laser besitzt 
eine Wellenlänge von $\SI{980}{\nano\meter}$. Der ermittelte Wert scheint daher 
etwas hoch, allerdings ist das Verhalten der Eindringtiefe mit wachsender 
Wellenlänge nicht bekannt (sprich ob exponentiell, linear, etc.) und kann daher 
nur als möglicher Wert bewertet werden. Jedoch ist zu erkennen, dass die Verwendung 
einer Quelle oder eines Lasers durchaus einen Einfluss auf die Charge Collection 
Efficiency hat. Das hat den Ursprung, dass der Laser bei einer 
Eindringtiefe von $a = \SI[parse-numbers=false]{174.43 \pm 5.8 \times 10^{-5}}{\micro\meter}$ 
in einen $\SI{300}{\nano\meter}$ dicken Laser, seine Energie vollständig 
deponieren kann. Bei der Quelle allerdings spielen statistische Fluktuationen 
auch eine Rolle. Wie bereits in der Theorie erwähnt, produzieren die ionisierenden 
Teilchen aus dem Beta-Zerfall Sekundärteilchen, welche eine Wahrscheinlichkeit 
besitzten, den Detektor verlassen zu können. Dies Beeinflusst die Charge Collection 
Efficiency, weswegen diese für die Quelle flacher ansteigt, als für den Laser. \par \medskip 

Die Darstellung der Kanäle pro Cluster und die Cluster pro Event passen gut 
zueinander. Sie zeigt, dass pro Event vermehrt wenige Cluster auftreten und drei 
oder mehr Cluster kaum eine Wahrscheinlichkeit besitzen. Des Weiteren sind vermehrt 
wenige Kanäle in einem Cluster zusammengefasst. Es gibt kaum Cluster mit vier oder 
mehr Kanäle.\par 
Die Bestimmung des Energiespektrums spiegelt weitesgehend die Erwartung wieder. Wie 
bereits beschrieben wird eine Faltung einer Gaußverteilung mit einer Landauverteilung 
erwartet mit einem Peak bei kleineren Energien und einem Mittelwert bei höheren Energien.
Der gemessene Mittelwert und MPV spiegeln diesen Sachverhalt wieder. Laut 
Anleitung gilt für $^{90}\text{Sr}$ eine mittlere deponierte Energie in Silizium 
von $\bar{\text{E}} = \SI{3.88}{\mega\electronvolt\per\centi\meter}$. Gerechnet auf die 
tatsächliche Dicke des Sensors ergibt dies eine mittlere Energiedeposition 
von $\bar{\text{E}} = \SI{116.4}{\mega\electronvolt}$ pro $\SI{300}{\centi\meter}$. 
Gemessen wurde eine mittlere Energiedeposition von 
$\bar{\text{E}} = \SI{157.61}{\kilo\electronvolt}$. Dies weicht um $\SI{35}{\percent}$
von dem erwarteten Ergebniss ab. Hierbei sind statistische Fluktuationen zu 
beachten. Des Weiteren macht es einen großen Einfluss, welcher Bereich 
des mittels Kalibrierungsfunktion überführten Spektrums für die Berechnung 
verwendet wurde. Hier wurde nur der Bereich der Messwerte der Kalibrierungsfunktion 
selber genommen, da der volle Messbereich den Mittelwert weiter zu hohen Energien 
geführt hätte. 