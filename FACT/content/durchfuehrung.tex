\section{Durchführung}
\label{sec:durch}
First the data selection was implemented to only include events which have a minimum of $\SI{80}{\percent}$CL.
Only the events with a angular distance to the source position of $\theta^2 \leq \num{0.025}$ were taken into consideration.
Afterwards the theta-squared plot was made and the detector significance was calculated.
Then, the migration matrix of the random-forest-regressor with the estimated- and true-gamma energies in an energy range from $\SI{500}{\giga\electronvolt}$ to $\SI{15}{\tera\electronvolt}$ was created.

Afterwards the unfolding for the crab nebula measurement with the naive SVD and the poisson-likelihood method was done.
At last the calculated flux for the unfolded data was compared to the HERA and MAGIC measurements.
