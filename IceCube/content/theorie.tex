\section{Theoretische Grundlagen}
\label{sec:theorie}
Kosmische Strahlung besteht hautps\"achlich aus hochenergetischen Protonen, schweren Kernen, Myonen und Neutrinos. Die Komposition der geladenen kosmischen Strahlung h\"angt dabei vom Energiebereich ab. Es k\"onnen dabei Energien bis zu $10^{20} \si{\electronvolt}$ erreicht werden. Die Energieverteilung folgt approximal einem Potenzgesetz der Form
\begin{equation*}
  \frac{\symup{d}\Phi}{\symup{d}E} = \Phi_0 E^{\gamma}
\end{equation*}

wobei $\gamma$ der spektrale Index von etwa $-2.7$ f\"ur geladene Teilchen ist.
Astrophysikalische Neutrinos stammen aus Quellen die auch Hadronen beschleunigen. Da Neutrinos einen sehr kleinen Wirkungsquerschnitt besitzen, durchdringen sie selbst dichte Staubwolken, welche f\"ur Photen ein Hindernis sind. Au\ss erdem werden Neutrinos nicht durch Magnetfelder abgelenkt und zeigen somit auf die Quelle und k\"onne so Informationen \"uber das innere der Quelle liefern.
Aufgrund von galaktischen Magnetfeldern ist es aber bislang nicht gelungen die Quellen der kosmischen Strahlung zu bestimmen.
Wenn zur Beschleunigung eine Art Sto\ss beschleunigung angenommen wird, also eine Art Fermibeschleunigung f\"ur Neutrinos, f\"uhrt dies auf ein Potenzgesetz f\"ur den Neutrinofluss mit spektralen Index von $\gamma \approx -2$.
Nun k\"onnen Neutrinos und Myonen auch aus Wechselwirkungen in der Atmosph\"are stammen, wo sie Zerfallsprodukte von Pionen und Kaonen sind. Da diese Mesonen eine vergleichsweise lange Lebensdauer haben verlieren sie vor ihrem Zerfall schon an Energie wodurch sich das Energiespektum einem Potenzgesetz proportional zu $E^{-3.7}$ gleicht. Die so entstandenen Neutrinos und Myonen nennt man konventionelle Neutrinos bzw. Myonen.
Andererseits gibt es sogenannte prompte Neutrinos, welche entstehen wenn in hochenergetischen Wechselwirkungen kurzlebige schwere Hadronen erzeugt werden und ohne nennenswerten Energieverlust wieder zerfallen. Aus den (semi-)leptonischen Zerf\"allen stammen Neutrinos und Myonen welche das  Energoespektrum der kosmischen Strahlung erben.

\subsection{Messung von Neutrinos mit IceCube}
Zur Messung wird die Analysetechnik "starting events" verwendet. Hierbei wird die \"au\ss ersten Detektorabschnitte als Veto verwendet werden um atmosph\"arische Myonen zu verwerfen.
Alle Neutrinoflavor werden hierbei ber\"ucksichtigt. Demnach sind die beiden gr\"o\ss ten Beitr\"age der Ereignisse aus dem neutralen Strom und der Elektron- bzw. Tauneutrinowechselwirkungen mittels des geladenen Stroms. Diese Ereignisse werden \"uber Kaskaden im Detektor sichtbar und haben eine gute Energieaufl\"osung, jedoch eine schlechte Winkelaufl\"osung.

Myonen die durch den kompletten Detektor propagieren haben aufgrund der niedrigen Energiedeposition eine schlechtere Energieaufl\"osung, ihre Spur kann jedoch gut rekonstruiert werden.
Die Erde kann als Schild f\"ur atmosph\"arische Myonen verwendet werden, da diese durch sie zu gr\"o\ss ten Teilen absorbiert werden. Myonen die von unten eintreffen, m\"ussen also aus Neutrinowechselwirkungen stammen.
Auf den Zenitwinkel kann deswegen ein Schnitt gesetzt werden um Myonneutrinos und Myonen getrennt werden. Dadurch kann das Signal-Untergrund Verh\"altnis von $1:\num{1e6}$ auf $1:\num{1e3}$ verbessert werden. 
