\section{Theoretische Grundlagen}
\label{sec:theorie}
Die geladene kosmische Strahlung besteht haupts\"achlich aus hochenergetischen Protonen, Helium und schweren Kernen.
Die Komposition der geladenen kosmischen Strahlung h\"angt dabei vom Energiebereich ab.
Es k\"onnen dabei Energien bis zu $10^{20} \si{\electronvolt}$ erreicht werden.
Die Energieverteilung folgt approximal einem Potenzgesetz der Form
\begin{equation*}
  \frac{\symup{d}\Phi}{\symup{d}E} = \Phi_0 E^{\gamma}
\end{equation*}

wobei $\gamma$ der spektrale Index von etwa $-2.7$ f\"ur geladene Teilchen ist.

Astrophysikalische Neutrinos stammen aus Quellen die auch Hadronen beschleunigen.
Diese Neutrinos  besitzen einen sehr kleinen Wirkungsquerschnitt, durchdringen sie selbst dichte Staubwolken, welche f\"ur Photonen ein Hindernis sind.
Diese Neutrinos werden nicht durch Magnetfelder abgelenkt und zeigen somit direkt auf die Quelle.
Sie k\"onnen so direkte Informationen \"uber das Innere der Quelle liefern.
Anders für höherenergetische kosmische Strahlung, diese können durch Magnetfelder abgelenkt werden.
Die Rekonstruktion auf eine spezifische Punktquelle ist daher sehr schwierig bis unmöglich.
Wenn zur Beschleunigung eine Art Sto\ss beschleunigung angenommen wird, also eine Art Fermibeschleunigung f\"ur die geladenen Teilchen, f\"uhrt dies auf ein Potenzgesetz für den Fluss, den die Neutrinos mit spektralen Index von $\gamma \approx -2$ erben.

Nun k\"onnen Neutrinos und Myonen auch aus Wechselwirkungen der kosmischen Strahlung mit der Atmosph\"are stammen, wo sie Zerfallsprodukte von Pionen und Kaonen sind.
 Da diese Mesonen eine vergleichsweise lange Lebensdauer haben verlieren sie vor ihrem Zerfall schon an Energie wodurch sich das Energiespektum einem Potenzgesetz proportional zu $E^{-3.7}$ gleicht.
 Die so entstandenen Neutrinos und Myonen nennt man konventionelle Neutrinos bzw. Myonen.
Andererseits gibt es sogenannte prompte Neutrinos, welche entstehen wenn in hochenergetischen Wechselwirkungen kurzlebige schwere Hadronen erzeugt werden und ohne nennenswerten Energieverlust wieder zerfallen.
Aus den (semi-)leptonischen Zerf\"allen stammen Neutrinos und Myonen welche das  Energiespektrum der kosmischen Strahlung erben.

\subsection{Messung von Neutrinos mit IceCube}
Eine mögliche Analysemethode verwendet sogenannte "starting events".
Hierbei werden die \"au\ss ersten Detektorabschnitte als Veto verwendet um atmosph\"arische Myonen zu verwerfen.
Alle Neutrinoflavor werden hierbei ber\"ucksichtigt.
Demnach sind die beiden gr\"o\ss ten Beitr\"age der Ereignisse aus dem neutralen Strom und der Elektron- bzw. Tauneutrinowechselwirkungen mittels des geladenen Stroms.
Diese Ereignisse werden \"uber Kaskaden im Detektor sichtbar und haben eine gute Energieaufl\"osung, jedoch eine schlechte Winkelaufl\"osung.

Myonen die durch den kompletten Detektor propagieren haben aufgrund der niedrigen Energiedeposition eine schlechtere Energieaufl\"osung, ihre Spur kann jedoch gut rekonstruiert werden.
Die Erde kann als Schild f\"ur atmosph\"arische Myonen verwendet werden, da diese durch sie zu gr\"o\ss ten Teilen absorbiert werden.
Myonen die von unten eintreffen, m\"ussen also aus Neutrinowechselwirkungen aus der Nähe des Detektors stammen.
Auf den Zenitwinkel kann deswegen ein Schnitt gesetzt werden um atmosphärische Myonneutrinos und Myonen zu trennen.
Der Schnitt auf den Zenitwinkel stellt eine alternative Analysemethode dar, die ohne "starting events" auskommt.
Dadurch ist das effektive Detektorvolumen im Vergleich zu der Analysemethode der "starting events" verbessert.
Dadurch kann das Signal-Untergrund Verh\"altnis von $1:10^6$ auf $1:10^3$ verbessert werden.
