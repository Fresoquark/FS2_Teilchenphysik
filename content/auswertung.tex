\section{Auswertung}
\label{sec:auswertung}
\subsection{invariante Masse der B-Mesonen in simulierten Daten}

Zu beginn haben wir die \texttt{\.root}-Dateien der simulierten Daten eingelesen und uns die mit den features vertraut gemacht.
Das Ziel im ersten Teil ist, die invariante Masse der B-Mesonen zu bestimmen. Da dies nicht direkt funktioniert muss dies \"uber die Tochterteilchen getan werden.
Wir betrachten hier ausschie\ss lich den Zerfall
\begin{equation}
  B^{\pm} \to \symup{K}^{\pm} \symup{K}^{+} \symup{K}^{-}\,.
\end{equation}
Um die invariante Masse zu bestimmen verwenden wir die Beziehung aus der speziellen Relativit\"atstheorie
\begin{equation}
  \symup{E}^2 = \symup{p}^2 + \symup{m}^2\,,
  \label{eqn:relativityEQ}
\end{equation}
welche Energie, Masse und Impuls verkn\"upft.

Aus den Daten entnehmen wir die Dreierimpulse der Tochterteilchen.
Diese plotten wir zun\"achst um einen Crosscheck zu machen ob die Daten auch Sinn ergeben.

% hier die Plots der impulse fuer x, y, z einbinden

Es ist zu erkennen, dass die Teilchen stark in z-Richtung geboostet sind, was auch zu erwarten ist bei B-Mesonen.

Um nun die invariante Masse zu berechnen, wird zun\"achst die Energie der B-Mesonen bestimmt.
\begin{equation}
  \symup{E}(B^{\pm}) =
  \sqrt{\left(\sum_{i = 1}^{3} \vec{\symup{p}}_i\right)^2 + \left(\sum_{i = 1}^{3} \symup{m}_i\right)^2}
\end{equation}

F\"ur die Massen wird die Massenhypothese der Kaonen eingesetzt, da dies der interessante Endzustand ist.

Mit der nun berechneten Energie kann durch umstellen von Gleichung \eqref{eqn:relativityEQ} auf den Impuls und damit auch auf das Betragsquadrat der B-Mesonen geschlossen werden.

In Abbildung \ref{fig:invMassB} liegt der Massenpeak bei etwa $\SI{5280}{\mega\electronvolt}$ was sehr eng an dem Wert des PDG liegt.

Dieser Peak ist so scharf, da es sich hier im simulierte Daten handelt. In Wirklichkeit sollte der peak etas verschmierter sein.

\subsection{invariante Masse der B-Mesonen in echten Daten}
Als n\"achstes wird die invariante Masse der echten B-Mesonen rekonstruiert.
Hierzu wird zun\"achst eine Vorselektion durchgef\"uhrt um nur den oben genannten Endzustand zu verwenden.
Dazu verwenden die folgenden Schnitte.
\begin{enumerate}
  \item \texttt{H1\_isMuon} = False
  \item \texttt{H1\_ProbPi} < 0.5
  \item \texttt{H1\_ProbK} > 0.5
\end{enumerate}

Diese Schnitte werden analog auch f\"ur Tochterteilchen 2 und 3 angewandt.
Die Verteilungen der Wahrscheinlichkeiten ob ein Finalstate Teilchen ein Kaon oder Pion ist haben wir geplottet um die Schnitte auf \texttt{H1\_ProbK} und \texttt{H1\_ProbPi} eventuell etwas sch\"arfer zu machen.
Die hatte aber nur zur Folfe, dass wir sehr viel Statistik Im Signal verloren haben und noch ziemlcih viel Hintergrund hatten. Deswegen haben wir die Schnitte wie oben belassen.

Anschlie\ss end haben wir wie schon bei den simulierten Daten die invariante Masse der B-Mesonen berechnet. Dieser ist in Abbildung \ref{fig:realBMass} dargestellt. Um den Massenplot f\"ur die $B^{-}$ von den $B^{+}$-Mesonen zu separieren haben wir die Ladung der Tochterteilchen multipliziert. Wenn das Produkt $+1$ war haben wir das Ereignis zu den $B^{-}$ gez\"ahlt, sonst zu den $B^{+}$.
