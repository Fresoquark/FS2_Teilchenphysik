\section{Diskussion}
\label{sec:discussion}

Beim Vergleich der rekonstruierten invarianten Massen für den simulierten Datensatz 
mit den echten Daten ist zu erkennen, dass der bei der Verwendung der echten Daten 
ein größerer Bereich mit Einträgen versehen ist. Das liegt daran, dass der Untergrund 
die Analyse deutlich beeinflusst. Im Zusammenhang damit ist auch die Entscheidung, wo 
die Grenzen für die Cuts gesetzt werden, relevant. Während der Analyse, wurde 
festgestellt, dass engere Grenzen für die Wahrscheinlichkeiten eines Pions oder 
Kaons zu deutlichen Signalverlust führt. Das bedeutet, dass die Entscheidung der 
gewählten Grenzen somit eine Abwägung sein muss, zwischen Verlust von Statistik 
und reineren Events. Die Position des Peaks entspricht sowohl bei der Verwendung 
echter Daten als auch bei den simulierten Daten einer sinnvollen Position entsprechend 
der $B$-Meson Masse des PDGs \ref{tab:PDGMassen}. \par 

Die Bestimmung der globalen $CP$-Asymmetrie liefert eine Signifikanz von 
$3.11\sigma$ mit Berücksichtigung der Unsicherheit durch 
Produktionsasymmetrie. In der Physik kann ab $5\sigma$ von einer Entdeckung 
gesprochen werden. Somit ist in dem verwendeten Datensatz mit dieser 
Selektion keine globale $CP$-Verletzung gefunden worden. Zudem wurde nur die 
Unsicherheit durch die Produktionsasymmetrie berücksichtigt. Weitere Effekte 
und Unsicherheiten die einen Einfluss haben können, wie Detektoreffekte und 
Unsicherheiten in der Luminosität oder ähnliches werden nicht berücksichtigt. 
Die Inkludierung dieser könnte die Signifikanz weiter verringern. Das LHCb 
Papier \ref{paper}, welches den gleichen Datensatz verwendet hat, berücksichtigt 
unter Anderem systematische Unsicherheiten in der Trigger Asymmetrie und 
in der Akzeptanzkorrektur. \par 

Die Darstellung der Dalitz-Plots liefert eine Aussage darüber, dass charmonium 
Zwischenresonanzen in dem verwendeten Datensatz vorhanden sind. Diese müssen 
für eine reinere Analyse entfernt werden. Des Weiteren ist es sinnvoll, 
Massenverteilungen zu betrachten, da durch die Messung der Massenanteile 
außerhalb der Signalregion Abschätzungen auf den Anteil im kombinatorischen 
Untergrund im Signalbereich gemacht werden können.\par 

Die Betrachtung der lokalen $CP$-Asymmetrie liefert eine Signifikanz 
von $4.07\sigma$ unter Berücksichtigung der Unsicherheit durch die 
Produktionsasymmetrie Asymmetrie. Auch hier kann nicht von einer Entdeckung 
gesprochen werden, aber es zeigt sich, dass die Suche nach lokaler $CP$-Verletzung 
größere Hinweise geben kann. Daher ist es sinnvoll, mehrere Zusammenhängende 
Gebiete auf lokale $CP$-Verletzung zu untersuchen. \par 

Der Vergleich der Anzahl der produzierten $B^-$-Mesonen und $B^+$-Mesonen 
verdeutlicht die Asymmetrie erneut und veranschaulicht eine Bevorzugung für die 
Produktion der $B^-$-Mesonen. \par 

In dieser Analyse wurde lediglich der Zerfall in drei Tocherkaonen untersucht, da 
dies der Kanal mit dem geringsten Untergrund ist. Bei der Betrachtung von Pionen 
im Endzustand tritt ein größerer Untergrund auf. Die Anayse aller möglichen 
Zerfallskanäle und deren Kombination könnten weitere signifikante Hinweise und 
Informationen auf $CP$-Verletzung liefern.