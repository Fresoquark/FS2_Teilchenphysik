\section{Der LHCb Detektor}
\label{sec:lhcb}

Der LHCb Detektor \cite{lhcb} ist stationiert am Large Hadron Collider (LHC) in
der Schweiz an der Forschungseinrichtung der europäischen
Organisation für Kernforschung names CERN. Der Detektor ist ein einarmiges
Vorwärtsspektrometer, konzipiert für die Messung von $B$-Mesonen, da diese
häuptsächlich in Vorwärtsrichtung produziert werden. In einem rechtshändigen
Koordinatensystem verläuft die $z$-Achse entlang der Strahlrichung und die
$y$-Achse entlang der vertikalen Achse. \par

Zur Rekonstuktion der Teilchenspur wird ein Siliziumstreifen-Vertex-Detektor
namens VELO oder auch Vertex-Locator verwendet. Dieser misst mindestens drei
Punkte jeder Teilchenspur und ist somit dazu in der Lage, den Primärvertex als
auch weitere sekundäre Vertices zu bestimmen. Durch die hohe Lebensdauer der beauty-quarks
sind die $B$-Mesonen in der Lage eine endliche Strecke im Detektor zurückzulegen,
bevor sie in ihre Zerfallsprodukte zerfallen. Der Vertex Detektor ist somit
essentiell für die Bestimmung dieser Vertices. \par
Vor dem Dipolmagneten befindet sich
der großflächige, vierlagige Siliziumstreifendetektor TT. Dort werden die durch den
Magneten gekrümmten Teilchenbahnen gemessen. Der Dipolmagnet erzeugt eine
integrierte Magnetfeldstärke von $\SI{4}{\milli\tesla}$ und wird regelmäßig
umgepolt um den Einfluss von systematischen Unsicherheiten zu verringern. Auf Grund
der Umpolung werden in diesem Versuch zwei verschiedene Datensätze verwendet, einer
mit positiver Polung und einer mit umgekehrter. \par
Hinter dem Dipolmagneten gibt es drei weitere Spurdetektoren T1, T2 und T3, welche
aus Siliziumstreifendetektoren und Driftröhren bestehen. Die Röhren sind mit
Argon und Kohlenstoffdioxid gefüllt, welche von den durchquerenden Teilchen
ionisiert werden. Diese Ionen erzeugen ein Signal nach dem durchqueren der Röhre. \par

Ein wichtiges Element des LHCb Detektors sind die Ring-Imaging Cherenkov Detektoren (RICH).
Sie werden verwendet um Informationen über die Art des Teilchens zu erhalten. Teilchen,
die das Material des RICH Detektors durchqueren erzeugen Cherenkov-Licht, wenn
ihre Geschwindigkeit höher ist, als die Lichtgeschwindigkeit in dem Material. Die
abgestrahlten Photonen werden unter ihren Abstrahlungswinkeln rekonstruiert, welche
von der Teilchengeschwindigkeit abhängen. Durch die Bestimmung der Geschwindigkeit
zusammen mit den Impulsinformationen kann die Masse der Teilchen und somit die
Teilchenart bestimmt werden.\par

Im Kalorimetersystem, bestehend aus einem Scintillating Pad (SPD), einem
Preshower Detektor (PS), dem elektromagnetischen Kalorimeter (ECAL) und
dem hadronischen Kalorimeter (HCAL) deponieren die meisten Teilchen je nach Teilchenart
ihre Energie. Myonen sind minimal ionisierende Teilchen und deponieren somit kaum
Energie in den Kalorimetern. Dafür sind am Ende des Detektors Myonkammern installiert. \par

Das Trigger System besteht aus einer Hardware-Implementation, welches Informationen aus
den Kalorimetern und den Myonkammern verwendet und einer Software-Implementation. Im
Rahmen der Software-Implementation kann das Ereignis vollständig rekonstruiert werden.
Die Trigger sind ein wichtiger Bestandteil jedes Experiments, da die große Menge an
Daten, die produziert wird, auf die interessanten Ereignisse reduziert werden muss,
da nicht genug Speicherplatz vorhanden ist, um jedes Ereignis abzuspeichern. Das Trigger
System dient somit zur Vorselektion. \par

Die hier verwendeten Daten wurden 2011 am LHCb Detekor aufgezeichnet. Die Datenmenge
korrespondiert zu einer integrierten Luminosität von $\SI{434}{\pico\barn}^{-1}$ für
die up-Position der Magnetfeldpolung und einer integrierten Luminosität von
$\SI{584}{\pico\barn}^{-1}$ für die down-Position.
