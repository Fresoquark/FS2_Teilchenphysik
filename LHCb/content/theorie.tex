\section{CP-Verletzung}

Der Begriff $CP$-Verletzung beschreibt die Verletzung der Symmetrie unter Ladungsumkehr $C$ und unter
Raumspiegelung oder auch Parität $P$. Operationen der Ladungsumkehr wandeln Teilchen in ihr
Antiteilchen um, da alle inneren Quantanzahlen konjugiert werden. Wird die Ladungskonjugation somit
zweimal angewandt, ergibt sich erneut das Ausgangsteilchen. Operationen unter Raumspiegelung ändern
die Händigkeit der Raumkomponenten. Die starke und die elektromagnetische Wechselwirkungen bewahren
die Symmetrie unter $C$-, $P$- und $CP$-Operationen. In der schwachen Wechselwirkung wird sowohl die
Ladungsumkehr als auch die Parität stark verletzt, die $CP$-Symmetrie ist allerdings in den meisten
schwachen Wechselwirkungen erhalten. In selten Zerfällen allerdings, wie beispielsweise Kaon
oder $B$-Meson Zerfällen, kann eine Verletzung der $CP$-Symmetrie nachgewiesen werden.  \par

Im Standard Modell wird die $CP$-Verletzung durch eine komplexe Phase in der CKM-Matrix
ausgedrückt. Bei der Erweiterung des Standard Models mit den Majorana Massentermen für die
Neutrinos finden sich drei komplexe Phasen in der Mischungsmatrix für Leptonen. \par

Eine Erhaltung der $CP$-Symmetrie würde bedeuten, dass es keine Bevorzugung zwischen Materie und
Antimaterie gibt. In unserem heutigen Universum lässt sich jedoch eine deutliche Asymmetrie
untersuchen, die einen größeren Materieanteil aufweist.
%Die $CP$-Verletzung, die in den selten Zerfälle gemessen wird, spiegelt lediglich einen Bruchteil der beobachteten Materie-Antimaterie-Asymmetrie wieder.
Die $CP$-Verletzung, die in den selten Zerfälle gemessen wird, kann lediglich einen Bruchteil der beobachteten Materie-Antimaterie-Asymmetrie widerspiegeln.
Daher muss es weitere $CP$-Verletzende Mechanismen geben, die noch unentdeckt sind. Die
Messungen der $CP$-Verletzungen sind eng verknüpft mit neuer Physik, da viele Modelle zur neuen
Physik neue Quellen dieser Verletzung diskutieren. \par

Zwei wichtige Experimente zur $CP$-Verletzung sind unter anderem Belle und LHCb. In diesem
Versuch werden Daten verwendet, welche am LHCb Experiment aufgenommen wurden, weswegen
ausschließlich dieses Experiment diskutiert wird. Das Experiment ist so konzipiert, dass
$B$-Mesonen äußerst präzise gemessen werden können. \par

\section{Der \textbf{\textit{B}}-Meson Zerfall und Zwischenresonanzen}

Betrachtet werden die Zerfälle von $B$-Mesonen oder genauer:

\begin{align}
    B^{\pm} \to h^{\pm} h^{\pm}  h^{\mp} \, .
\end{align}

Die Hadronen $h^\pm$ können dabei sowohl Pionen als auch Kaonen sein. Wie bereits diskutiert, kann die
$CP$-Verletzung in der schwachen Wechselwirkung nachgewiesen werden. Die untersuchten Zerfälle
werden somit über diese vermittelt. Die Verletzung äußert sich in der Anzahl der gemessenen
$B^+$ und $B^-$-Zerfällen und die Asymmetrie zwischen den beiden Variablen. Dabei muss beachtet
werden, dass die am Large Hadron Collider (LHC) produzierten Teilchen aus Proton-Proton
Kollisionen stammen. Somit ist kein Materie-Antimaterie gleicher Anfangszustand gegeben
und Nebeneffekte, die eine scheinbare Asymmetrie verursachen, müssen berücksichtigt werden. \par

Wichtig ist zudem die Betrachtung von Zwischenresonanzen. Die $B$-Mesonen selber können nicht
gemessen werden, sondern müssen über ihre Zerfallsprodukte identifiziert werden. Dabei muss
beachtet werden, dass Zerfälle mit Zwischenresonanzen gleiche Endzustandsteilchen bilden
können. Zwischenresonanzen mit einem charm-Quark sind die am häufigsten vorkommenden
Resonanzen in $B$-Zerfällen. Die dabei relevanten Teilchen sind unter anderem das
$D$-Meson, das $J/\psi$-Meson oder auch das $\Chi$-Meson. Für die spätere Darstellung der
Dreikörper-Zerfälle werden die Massen aller möglichen Teilchen benötigt. Die Massen sind in
Tabellle \ref{tab:PDGMassen} aufgelistet.

\begin{table}
    \centering
    \caption{Aus dem PDG \cite{PDG} entnommenen Massen für die involvierten Teilchen im
        Dreikörper-$B$-Meson Zerfall mit ihrem zugehörigen Quarkinhalt. Verwendet werden natürliche
        Einheiten mit $c = 1$.}
    \label{tab:PDGMassen}
    \begin{tabular}{c c c}
    \toprule
    {Teilchen} & {Quarkinhalt} & {Masse $\SI{}{\mega\electronvolt}$} \\
    \midrule
    {$B$-Meson $B^{\pm}$} & {$B^+$: $u\bar{b}$,$B^-$: $\bar{u}b$} & 5279.32 \\
    {Kaon $K^{\pm}$} & {$K^+$: $u\bar{s}$,$K^-$: $\bar{u}s$} & 493.68 \\
    {$D^0$-Meson} & {$c\bar{u}$} & 1864.83 \\
    {$J/\psi$-Meson} &{$c\bar{c}$} & 3096.90 \\
    {$\chi_{c0}\left(1P\right)$- Meson} & {$c\bar{c}$}& 3414.71 \\
    \bottomrule
    \end{tabular}
\end{table}
